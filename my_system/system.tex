\documentclass[11pt,a4paper,oneside]{report}
\usepackage{color}
\usepackage{amsmath}
\usepackage{arev}
\usepackage{kpfonts}
\usepackage[utf8]{inputenc}



\newcommand*\Hs[1]{\ensuremath{{\color{blue} #1}{\color{red}\varheartsuit}}}
\newcommand*\Ss[1]{\ensuremath{{\color{blue} #1}{\color{black}\spadesuit}}}
\newcommand*\Ds[1]{\ensuremath{{\color{blue} #1}{\color{red}\vardiamondsuit}}}
\newcommand*\Cs[1]{\ensuremath{{\color{blue} #1}{\color{black}\clubsuit}}}
\newcommand*\NT[1]{{\color{blue} #1}{\color{black}\textsc{ba}}}
\newcommand{\tab}{\hspace*{15em}}
\newcommand*\Hand[4]{\\\tab\Ss{}#1\\\tab\Hs{}#2\\\tab\Ds{}#3\\\tab\Cs{}#4\\}

\title{System}
\author{Krystian Dowolski}
\begin{document}

\maketitle
\section*{1BA}
Słabe BA ma wiele zalet, jak i wad, choć w mojej opinii te pierwsze zwycieżają.
Może służyć zarówno szybkiemu opisowi ręki jak i mieć charakter blokujący:
\Hand{Kx}{Wx}{A10xxx}{KW9x}
Otwieramy \NT{1}, bardzo możliwe, że przeciwnikom idzie coś w stary za 110,
ewentualna wpadka za 100 powinna więc być opłacalna, po kontrze mamy na co uciekać.\\
Problemem jest tylko sytuacja, w której partner jest bardzo słaby, nie trafimy w co uciec
i w dodatku nas skontrują. Choć takie sytuacje nie zdarzają się często, preferuję
otwieranie \NT{1} z bardzo różnymi rodzajami rąk - 5332 z dowolną! piątką, 6M322, 4441
z singlowym asem czy 5M422 z wartościami w dublach. Innymi słowy, otwieramy \NT{1}, gdy
nie mamy wyraźnych przeciwskazań, przeciwko graniu BA. Ułatwi to zdecydownie licytację
po otwarciach w kolor.\\
Przed partią 11+--16-, po od 12+\\

\subsection*{1BA-\Cs{2}}
W większości wypadków jest to stayman, niektórzy grają puppet staymanem lub staymanem
rozkładowym. Wydaje się, że przy otwarciach BA ze starszą piątką, konieczne jest granie
puppet staymanem. Jednak takie ustalenie nakazuje pasować z ręką typu:
\Hand{KWxx}{W10xxx}{xx}{xx}
co może się źle skończyć. Druga sprawa jest taka, że często po staymanie i tak
gramy 2/3BA. Gdyby dało się go uczynić mniej informacyjnym, co przy odchyłach w otwieraniu
1BA często okaże się ważne, moglibyśmy wygrywać końcówki na nietrafionym wiście.
Dlatego pierwszą propozycją jest to, by stayman \textbf{nie przyrzekał starszej czwórki}.
Pozwala to wykorzystać więcej odzywek, a także (w moim odczuciu) jest mniej informacyjne.
Popatrzmy:\\\\
\begin{minipage}[c]{0.30\textwidth}
1BA -- \Cs{2}\\
\Ds{2} -- 2/3BA
\end{minipage}%
\begin{minipage}{0.30\textwidth}
1BA -- \Cs{2}\\
\Hs{2} -- 2/3BA
\end{minipage}%
\begin{minipage}{0.30\textwidth}
1BA -- \Cs{2}\\
\Ss{2} -- 2/3BA
\end{minipage}%
\\\\
W pierwszej sekwencji wiemy, że rozgrywający nie ma S4, nic nie wiemy o dziadku.
(Analogia do licytacji 1BA -- 3BA, w której wiemy to o ręce dziadka). Druga sekwencja
przekazuje niestety informację o kierach rozgrywającego, trzecia natomiast, w przeciwieństwie
do klasycznego staymana, nie mówi nic o tym kolorze, czyniąc wist kierowy bardziej atrakcyjnym -
być może w kolor naszego czterokartu.\\
Pozostaje jeszcze zagadnienie odnalezienia koloru 5-3, proponuję zastosować następujący
schemat odpowiedzi:\\\\
\Ds{2} -- brak \Hs{4} i \Ss{5}\\
\Hs{2} -- \Hs{4+}\\
\Ss{2} -- \Ss{5}\\
Dalsza licytacja jest klasyczna, z tym, że \Ss{2} oznacza po prostu co najmniej
inwit z 4 pikami. Klasyczne znaczenie tej odzywki (czyli inwit 5\Ss{}4\Hs{}) wsadzimy gdzie indziej.

\subsection*{1BA-\Ds{2}}
Prawie zwykły transfer, przy czym może być z czwórki przy pięciu pikach z boku w sile co najmniej inwitu.

\subsection*{1BA-\Ss{2}}
Z tym było trochę problemów. Chciałem wpakować w tę odzywkę słabą rękę na młodych, coś w stylu
\Hand{xx}{xx}{D109x}{DW8xx}
czyli coś, z czym nie chcemy zostać na 1BA, a także nie chcemy przesądzać grę w konkretny młody.
Oprócz tego musiał także istnieć możliwość sprzedania młodego w dowolnej sile. Proponuję takie
rozwiązanie:\\
\subsection*{\Ss{2}}
\begin{enumerate}
    \item wspomniana ręka
    \item 6+\Ds{} w dowolnej sile
\end{enumerate}
\subsection*{\Ss{2} -- ?}
2BA -- maksimum\\
\Cs{3} -- minimum trefle nie krótsze\\
\Ds{3} -- minimum kara dłuższe\\
\newpage
\noindent Oczywiście minimum w kontekście 6 kar u partnera. Ręka typu
\Hand{D109x}{K10x}{Axxx}{Kx}
raczej wystarcza na odpowiedź 2BA

\subsection*{2BA}
Słabe lub silne na treflach, licytacja analogiczna jak po \Cs{3} w znaczeniu transferu.

\subsection*{\Cs{3}}
Po prostu inwit na treflach

A gdzie klasyczny inwit? W takim wypadku przechodzimy przez staymana, a potem zapowiadamy 2BA
\end{document}
