\documentclass[12pt]{article}
\usepackage{bridge}


\title{System}
\author{Krystian Dowolski}
\begin{document}


\tableofcontents
\newpage

\section{Styl}
To kwestia uzgodnienia w parze. Nie wnikam w to, czy otwieracie od ładnych 13 czy brzydkich 10. Ani w to, czy wchodzicie z dobrym kolorem, z punktami, na wist czy tylko po to, by zrobić zamieszanie. Najważniejsze jest to, żeby wiedzieć, czego spodziewać się po partnerze.

\vspace*{0.7cm}
\hbadness10000
1)
\begin{minipage}[t]{0.25\textwidth}
    \Hand{KD}{10xxxx}{DWxx}{KW}
\end{minipage}%
2)
\begin{minipage}[t]{0.25\textwidth}
    \Hand{Axxxx}{Kxxx}{KWx}{x}
\end{minipage}%
3)
\begin{minipage}[t]{0.25\textwidth}
    \Hand{xx}{KD109xxx}{xxx}{x}
\end{minipage}%
\\\\
Otwieracie? Nie otwieracie? Nieważne, ważne, żeby partner wiedział i zgadzał się na to.
\newpage
\section{Otwarcie \texorpdfstring{\Cs[1]}{1C}}
\begin{itemize}
    \item 12–14PC lub 18+ – układ bezatutowy
    \item (14)15+  – 5t układ różny od 5(332)
\end{itemize}
\subsection{\texorpdfstring{\Cs[1] – ?}{1C - ?}}
\begin{itemize}
    \item \Ds[1] = negat, 0–7 na dowolnym lub 8–11 bez S4 i dobrej M6
    \item \Hs[1]/\Ss[] = 4+
    \item \NT[1] = zrównoważony, 8–10
    \item \Cs[2] = 5+\Cs, GF
    \item \Hs[2]/\Ss = 6+ bez A lub K bokiem, 5–8
    \item \NT[2] = zrównoważony, 11–12
    \item \Cs[3]/\Ds = dobry kolor 6+, 9–11
    \item \Ss[3] = transfer na 3\NT
\end{itemize}
\newpage
\subsection{\texorpdfstring{\Cs[1] – \Ds[1]}{1C – 1D}}
\begin{itemize}
    \item \Hs[1] = 3+\Hs, 12–14 lub 5\Hs332
    \item \Hs[1] = 3+\Ss, 12–14 lub 5\Ss332
    \item \NT[1] = zrównoważony, 18–21
    \item \Cs[2] = 5+\Cs, 15+
    \item \Ds[2] = skład dowolny, 22+ (acol)
    \item \Hs[2] = 6\Hs lub 5\Hs 4 z boku, 18+
    \item \Ss[2] = 6\Ss lub 5\Ss 4 z boku, 18+
    \item \NT[2] = zrównoważony, 22–23
    \item \Cs[3] = 6+\Cs, 18+
\end{itemize}
\subsection{\texorpdfstring{\Cs[1] – \Hs[1]/\Ss}{1C – 1H/S}}
\begin{itemize}
    \item \Ss[1] = 4+\Ss
    \item \NT[1] = zrównoważony, 12–14
    \item \Cs[2] = 5+\Cs, 15+
    \item \Ds[2] = Odwrotka, 3+\Hs/\Ss, 18+
    \item \Hs[2]/\Ss[] = 4+\Hs/\Ss, 12-14
    \item \NT[2] = bez 3\Hs/\Ss, 18+
\end{itemize}
\subsection{\texorpdfstring{\Cs[1] – \NT[1]}{1C – 1NT}}
% \begin{samepage}
\begin{itemize}
    \item \Cs[2] = 5+\Cs, 15+
    \item \Hs[2]/\Ss[] = 5+, 18+
\end{itemize}
% \end{samepage}

\newpage
\section{Otwarcie \texorpdfstring {\Ds[1]}{1D}}
\begin{itemize}
    \item 5+\Ds, 12–17
    \item 4\Ds, skład 4441 lub 4\Ds5\Cs
\end{itemize}
\subsection{\texorpdfstring{\Ds[1] – ?}{1D – ?}}
\begin{itemize}
    \item \Cs[2] = 5+\Cs, GF
    \item \Ds[2] = 3+\Ds, 10+
    \item \Hs[2]/\Ss[] = 6+ bez A lub K bokiem, 5–8
    \item \NT[2] = zrównoważony, 11–12
    \item \Cs[3] = dobry kolor 6+, 9–11
    \item \Ds[3] = 4\Ds, 5–8
    \item \Hs[3]/\Ss[] = splinter, forsuje do \Ds[4]
\end{itemize}

\newpage
\section{Otwarcie \texorpdfstring{\Hs[1]}{1H}}
\begin{itemize}
    \item 5+\Hs, 12–17 (z wyłączeniem składu 35(32) w sile 15–17)
\end{itemize}
\subsection{\texorpdfstring{\Hs[1] – ?}{1H – ?}}
\begin{itemize}
    \item \Cs[2] = 2+\Cs, GF
    \item \Ds[2] = 5+\Ds, GF
    \item \Hs[2] = 3+\Hs, 6-9 
    \item \Ss[2] = 6+ bez A lub K bokiem, 5–8
    \item \NT[2] = 3–4\Hs, 10–12 inv z fitem
    \item \Cs[3]/\Ds[] = dobry kolor 6+, brak fitu \Hs, 9–11
    \item \Hs[3] = 4+\Hs, blokujące
    \item \Ss[3]/\Cs[4]/\Ds[4] = splinter, forsuje do \Hs[4]
\end{itemize}

\newpage
\section{Otwarcie \texorpdfstring{\Ss[1]}{1S}}
\begin{itemize}
    \item 5+\Ss, 12–17
\end{itemize}
\subsection{\texorpdfstring{\Ss[1] – ?}{1S – ?}}
\begin{itemize}
    \item \Cs[2] = 2+\Cs, GF
    \item \Ds[2] = 5+\Ds, GF
    \item \Hs[2] = 5+\Hs, GF
    \item \Ss[2] = 3+\Ss, 6-9 
    \item \NT[2] = 3+\Ss, 10–12
    \item \Cs[3]/\Ds/\Hs[] = dobry kolor 6+, brak fitu \Ss, 9–11
    \item \Ss[3] = 4+\Ss, blokujące
    \item \Cs[4]/\Ds/\Hs[] = splinter, forsuje do \Ss[4]
\end{itemize}

\newpage
\section{Otwarcie \texorpdfstring{\NT[1]}{1NT}}
\begin{itemize}
    \item Skład zrównoważony, 15-17
\end{itemize}
\subsection{\texorpdfstring{\NT[1] – ?}{1NT – ?}}
\begin{itemize}
    \item \Cs[2] = pytanie o starszą czwórkę, 4\Hs4\Ss 0+ lub od 8
    \item \Ds[2]/\Hs[] = transfer, 5+\Hs/\Ss, 0+
    \item \Ss[2] = 6+\Cs, 0+
    \item \NT[2] = 6+\Ds, 0+
    \item \Cs[3]/\Ds[] = ktoś to w ogóle czyta [czyta czyta;), serio 12-18?:D], 12-18 
    \item \Hs[3]/\Ss[] = 1-3(45)+/31-(45)+, GF
\end{itemize}

\newpage
\section{Otwarcie \texorpdfstring{\Cs[2]}{2C}}
\begin{itemize}
    \item 6+\Cs, 11-14
    \item 5\Cs[] i 4\Hs/4\Ss[] ale \textbf{bez składu 4405}
\end{itemize}
\subsection{\texorpdfstring{\Cs[2] – ?}{1C – ?}}
\begin{itemize}
    \item \Ds[2] = pytanie o skład, S4 i 8+ lub dowolne 10+
    \item \Hs[2]/\Ss[] = 5+, NF
    \item \NT[2] = transfer na 3\Cs, dobry fit 0-9 albo 55 bez \Cs, 9-12
    \item \Cs[3] = transfer na 3\Ds, kara, SO albo 55 bez \Cs, GF
    \item \Ds[3]/\Hs/\Ss[] = 6+, inwit
\end{itemize}

\newpage
\section{Otwarcie \texorpdfstring{\Ds[2]}{2D}}
\begin{itemize}
    \item Sekcja dla Pana Macieja Boczara
    \item 6+\Hs[]/\Ss[], 6-10
\end{itemize}

\newpage
\section{Otwarcie \texorpdfstring{\Hs[2]}{2H}}
\begin{itemize}
    \item Sekcja dla Pana Macieja Grabca
\end{itemize}

\newpage
\section{Otwarcie \texorpdfstring{\NT[2]}{2NT}}
\begin{itemize}
    \item Sekcja dla pracowitej Paulinki
    \item Awww urocze \Hs \Hs \Hs  (chyba pierwszy raz w życiu, ktoś mnie nazwał 'pracowitą' ;))) Aż poszłam po książeczkę Jassema WJ 2020 i dzielnie wklepuję ;)
    \item 5+\Cs\Ds, 6–11 
\end{itemize}
\subsection{\texorpdfstring{\NT[2] – ?}{2NT – ?}}
\begin{itemize}
    \item \Cs[3]/\Ds[] = wybór lepszego koloru, do pasa
    \item \Hs[3] = sztuczne pytanie o skład
    \item \Ss[3] = sztuczny inwit do końcówki lub szlemika w kolor młodszy
    \item \NT[3] = do gry
    \item \Cs[4]/\Ds[] = blokujące
    \item \Hs[4]/\Ss[] = do gry
    \item \NT[4] = prośba o wybór lepszego młodszego, blokujące
    \item \Cs[5]/\Ds[5] = do gry
\end{itemize}
\subsubsection{\texorpdfstring{\NT[2] – \Hs[3] – ?}{2NT – 3H - ?}}
\begin{itemize}
    \item Odzywka \Hs[3] jest forsującym do końcówki pytaniem o układ. Dajemy ją z szansami na grę premiową.
    \item \NT[2] – \Hs[3]
  \\ ?
  \item \Ss[3] = krótkość \Ss[]
  \item \NT[3] = krótkość \Hs[]
  \item \Cs[4]/\Ds[] = licytowana 6, skład 1156(65)
  \item \Hs[4]/\Ss[] = licytowany renons
  \item Po odpowiedziach \Ss[3]/\NT[] odzywka \Cs[4]/\Ds[] silnie uzgadnia wskazany kolor i prosi o cuebid (dalej \NT[4] jest pytaniem o asy na wskazanym kolorze)
  \item Po odpowiedziach na poziomie 4, \NT[4] jest blackwoodem na 4 wartości
  \end{itemize}
\subsubsection{\texorpdfstring{\NT[2] – \Ss[3] – ?}{2NT – 3S - ?}}
\begin{itemize}
    \item Odzywka \Ss[3] jest inwitem do końcówki/szlemika. Otwierający licytuje obowiązkowo \NT[3] a odpowiadający wybiera kolor młodszy: \Cs[4]/\Ds[] są inwitem do końcówki w licytowany kolor, \Cs[5]/\Ds[] są inwitem do szlemika w ten kolor.
\end{itemize}

\newpage
\section{Otwarcie \texorpdfstring{\Cs[2]}{2C}}
\begin{itemize}
    \item 6+\Cs, 11-14
    \item 5\Cs[] i 4\Hs/4\Ss[] ale \textbf{bez składu 4405}
\end{itemize}
\subsection{\texorpdfstring{\Cs[2] – ?}{1C – ?}}
\begin{itemize}
    \item \Ds[2] = pytanie o skład, S4 i 8+ lub dowolne 10+
    \item \Hs[2]/\Ss[] = 5+, NF
    \item \NT[2] = transfer na 3\Cs, dobry fit 0-9 albo 55 bez \Cs, 9-12
    \item \Cs[3] = transfer na 3\Ds, kara, SO albo 55 bez \Cs, GF
    \item \Ds[3]/\Hs/\Ss[] = 6+, inwit
\end{itemize}

\newpage
\section{Otwarcie \texorpdfstring{\Cs[2]}{2C}}

\end{document}

