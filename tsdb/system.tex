\documentclass[12pt]{article}
\usepackage{color}
\usepackage{amsmath}
\usepackage[utf8]{inputenc}
\usepackage{fdsymbol}
\usepackage{hyperref}
\usepackage{titlesec}


% \newcommand*\Hh[0]{{\color{red}\varheartsuit}}
% \newcommand*\S{\color{black}\spadesuit}
% \newcommand*\D{\color{red}\vardiamondsuit}
% \newcommand*\C{\color{black}\clubsuit}
\titleformat*{\section}{\Huge\bfseries}
\titleformat*{\subsection}{\LARGE\bfseries}

\newcommand*\Hs[1][]{\ensuremath{{\color{blue} #1}{\color{red}\varheartsuit}}}
\newcommand*\Ss[1][]{\ensuremath{{\color{blue} #1}{\color{black}\spadesuit}}}
\newcommand*\Ds[1][]{\ensuremath{{\color{blue} #1}{\color{red}\vardiamondsuit}}}
\newcommand*\Cs[1][]{\ensuremath{{\color{blue} #1}{\color{black}\clubsuit}}}
\newcommand*\NT[1][]{{\color{blue} #1}{\color{black}\textsc{ba}}}
\newcommand{\tab}{\hspace*{15em}}
% \newcommand*\Handd[4]{\\\tab\Ss{}#1\\\tab\Hs{}#2\\\tab\Ds{}#3\\\tab\Cs{}#4\\}
\newcommand*\Hand[4]{\Ss#1\\\Hs#2\\\Ds#3\\\Cs#4\\}

\renewcommand\contentsname{Spis treści}

\title{System}
\author{Krystian Dowolski}
\begin{document}


\tableofcontents
\newpage

\section{Styl}
To kwestia uzgodnienia w parze. Nie wnikam w to, czy otwieracie od ładnych 13 czy brzydkich 10. Ani w to, czy wchodzicie z dobrym kolorem, z punktami, na wist czy tylko po to, by zrobić zamieszanie. Najważniejsze jest to, żeby wiedzieć, czego spodziewać się po partnerze.

\vspace*{0.7cm}
1)
\begin{minipage}[t]{0.30\textwidth}
    \Hand{KD}{10xxxx}{DWxx}{KW}
\end{minipage}%
2)
\begin{minipage}[t]{0.30\textwidth}
    \Hand{Axxxx}{Kxxx}{KWx}{x}
\end{minipage}%
3)
\begin{minipage}[t]{0.30\textwidth}
    \Hand{xx}{KD109xxx}{xxx}{x}
\end{minipage}%
\\\\
Otwieracie? Nie otwieracie? Nieważne, ważne, żeby partner wiedział i zgadzał się na to.
\newpage
\section{Otwarcie \Cs[1]}
\begin{itemize}
    \item 12-14PC lub 18+ – układ bezatutowy
    \item (14)15+  - 5t układ różny od 5(332)
\end{itemize}
\subsection{\Cs[1] - ?}
\begin{itemize}
    \item \Ds[1] = negat, 0-7 na dowolnym lub 8-11 bez S4 i dobrej M6
    \item \Hs[1]/\Ss = 4+
    \item \NT[1] = 8-10, zrównoważony
    \item \Cs[2] = 5+\Cs, GF
    \item \Hs[2]/\Ss = 5-8, 6+ bez A lub K bokiem
    \item \NT[2] = 11-12, zrównoważony
    \item \Cs[3]/\Ds = 9-11, 6+, dobry kolor
    \item \Ss[3] = transfer na 3\NT
\end{itemize}
\newpage
\subsection{\Cs[1] - \Ds[1]}
\begin{itemize}
    \item \Hs[1] = 3+\Hs, 12-14 lub 5\Hh 332
    \item \Ss[1] = 3+\Ss
    \item \NT[1] = 18-21, zrównoważony
    \item \Cs[2] = 5+\Cs,15+
    \item \Ds[2] = 22+ (acol)
    \item \Hs[2] = 6\Hs lub 5\Hs 4 z boku, 18+
    \item \Ss[2] = 6\Ss lub 5\Ss 4 z boku, 18+
    \item \NT[2] = 22-23, zrównoważony
    \item \Cs[3] = 6+\Cs,18+
\end{itemize}
\subsection{\Cs[1] - \Hs[1]/\Ss}
\begin{itemize}
    \item \Ss[1] = 4+\Ss{}
    \item \NT[1] = 12-14, zrównoważony
    \item \Cs[2] = 5+\Cs{},15+
    \item \Ds[2] = Odwrotka, 3+\Hs/\Ss, 18+
    \item \Hs[2]/\Ss = 4+\Hs/\Ss, 12-14
    \item \NT[2] = 18+, bez fitu
\end{itemize}
\subsection{\Cs[1] - \NT[1]}
% \begin{samepage}
\begin{itemize}
    \item \Cs[2] = 5+\Cs{},15+
    \item \Hs[2]/\Ss = 5+, silny trefl
\end{itemize}
% \end{samepage}

\newpage
\section{Otwarcie \Ds[1]}
\begin{itemize}
    \item 5+\Ds, 12-17
    \item 4\Ds, skład 4441 lub 4\Ds5\Cs
\end{itemize}
\subsection{\Ds[1] - ?}
\begin{itemize}
    \item \Cs[2] = 5+\Cs, GF
    \item \Ds[2] = 3+\Ds, 10+
    \item \Hs[2]/\Ss = 5-8, 6+ bez A lub K bokiem
    \item \NT[2] = 11-12, zrównoważony
    \item \Cs[3] = 9-11, 6+, dobry kolor
    \item \Ds[3] = 4\Ds, 5-8
    \item \Hs[3]/\Ss = splinter, forsuje do \Ds[4]
\end{itemize}
\end{document}
